\section{Larry's Race}

TJ IOI Inc.\@ has chosen Larry as their corporate representative at the local track and field competition! However, the competition has a very peculiar set of rules: if Larry would like to advertise TJ IOI Inc., he must compete in the race! To get Larry in shape, Devon has built a robot to chase Larry, traveling $ 100 $ meters in $ T $ seconds ($ 1 \leq T \leq 100,000 $).
\blank
There are $ N $ inputs to this problem ($ 1 \leq N \leq 100,000 $). Each consists of a distance $ A_i $ that Larry runs, where $ A_i $ is a multiple of $ 100 $ ($ 100 \leq A_i \leq 100,000 $), and the time $ B_i $ it took for him to run that distance ($ 1 \leq B_i \leq 100,000 $), determine whether Larry could outrun Devon's robot.
\blank
\noindent Note: if Devon's robot catches Larry exactly at the finish line, Larry did not outrun it.
\blank
\textbf{SHORT NAME:} \verb|race|
\blank
\textbf{INPUT FORMAT:}\\
The first line consists of two integers, $ N $ and $ T $.
The next $ N $ lines each contain an integer $ A_i $ representing a distance in meters ($ A_i $ is a multiple of $ 100 $), and a time $ B_i $ representing the time it took Larry to run that distance in seconds.
\blank
\textbf{OUTPUT FORMAT:}\\
For each input, if Larry outran Devon's robot, output ``SPEEDRACER'' (without quotes). Otherwise, output ``POTATO'' (without quotes).
\blank
\textbf{SAMPLE INPUT:}
\begin{verbatim}
3 22
1600 352
800 150
3200 840
\end{verbatim}
\textbf{SAMPLE OUTPUT:}
\begin{verbatim}
POTATO
SPEEDRACER
POTATO
\end{verbatim}
