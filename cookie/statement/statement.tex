\documentclass{article}
\usepackage[utf8]{inputenc}
\usepackage[margin=1in]{geometry}
\usepackage{titling}

\renewcommand\maketitlehooka{\null\mbox{}\vfill}
\renewcommand\maketitlehookd{\vfill\null}

\begin{document}

\newcommand{\blank}{\vskip 3mm}
\setlength\parindent{0pt}
\renewcommand\thesection{\Alph{section}}

\setcounter{section}{7}
\section{Cookie Baking}

Devon has $ N $ large piles of cookies ($ 1 \leq N \leq 10^5 $), where the $i$th pile ($ 1 \leq i \leq N $) contains $a_i$ cookies ($ 1 \leq a_i \leq 10^5$). Alex, on the other hand, wants to steal his cookies, but he brings along a different number of friends each time.

When he steals cookies, he wants to make sure he is able to split the cookies evenly among him and his $ X-1 $ friends ($ X $ people total). Alex always chooses $X$ to be prime, because he likes prime numbers. Since the door is located next to cookie pile $ 1 $, Alex wants to take cookies from the first possible pile (i.e. the minimum value of $i$), such that he can split the pile's cookies amongst $ X $ people evenly.

Since Devon is rich, whenever Alex takes cookies from a pile, he is able to restock the pile with exactly the same number of cookies. This means that the sizes of the cookie piles effectively do not change.

Unfortunately, Devon has caught on. To slow Alex down, he occasionally swaps two piles of cookies. Whenever Alex arrives, help him determine the best pile to take cookies from.

\blank
\textbf{INPUT/OUTPUT FORMAT:}\\
The first line contains $ N $ and $Q$ ($1 \leq Q \leq 10^5$).
The second line contains $N$ integers representing $a_i$.
The next $Q$ lines contains queries. Each of these lines will start with 'S' or 'T'.
\begin{itemize}
    \item If the line starts with 'S': two integers $x$, $y$ ($1 \leq x < y \leq 10^5$) will follow, denoting Devon swapping piles $x$ and $y$
    \item If the line starts with 'T': one integer $X$ ($1 \leq X \leq 10^5$, $X$ is prime) will follow. Output the minimum $i$ such that $X$ divides $a_i$, or output $-1$ if no such $ a_i $ exists. 
\end{itemize}
\blank
\textbf{SAMPLE INPUT:}
\begin{verbatim}
5 9
2 15 49 11 17
T 3
S 3 5
T 7
S 1 4
S 3 4
S 1 2
T 2
T 3
T 13
\end{verbatim}
\textbf{SAMPLE OUTPUT:}
\begin{verbatim}
2
5
3
1
-1
\end{verbatim}
\end{document}
