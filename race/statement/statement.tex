\documentclass{article}
\usepackage[utf8]{inputenc}
\usepackage[margin=1in]{geometry}
\usepackage{titling}

\renewcommand\maketitlehooka{\null\mbox{}\vfill}
\renewcommand\maketitlehookd{\vfill\null}

\begin{document}

\newcommand{\blank}{\vskip 3mm}
\setlength\parindent{0pt}
\renewcommand\thesection{\Alph{section}}

\setcounter{section}{0}
\section{Larry's Race}

Larry is attempting to join spring track! To get Larry in shape, Devon has built a robot to chase Larry, traveling $ 100 $ meters in $ T $ seconds ($ 1 \leq T \leq 100,000 $). Given $ N $ distances $ A_i $ that Larry runs and the time $ B_i $ it took for him to run that distance ($ 1 \leq N \leq 100,000 $, $ 100 \leq A_i \leq 100,000 $, $ 1 \leq B_i \leq 100,000 $), determine whether Larry could outrun Devon's robot. (Note: $ A_i $ will be a multiple of $ 100 $.)
\blank
\noindent Note: if Devon's robot catches Larry exactly at the finish line, Larry did not outrun it.
\blank
\textbf{INPUT FORMAT:}\\
The first line consists of two integers, $ N $ and $ T $.
The next $ N $ lines each contain an integer $ A_i $ representing a distance in meters ($ A_i $ is a multiple of $ 100 $), and a time $ B_i $ representing the time it took Larry to run that distance in seconds.
\blank
\textbf{OUTPUT FORMAT:}\\
For each input, if Larry outran Devon's robot, output "SPEEDRACER" (without quotes). Otherwise, output "POTATO" (without quotes).
\blank
\textbf{SAMPLE INPUT:}
\begin{verbatim}
3 22
1600 352
800 150
3200 840
\end{verbatim}
\textbf{SAMPLE OUTPUT:}
\begin{verbatim}
SPEEDRACER
POTATO
SPEEDRACER
\end{verbatim}
\end{document}
