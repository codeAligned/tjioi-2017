\documentclass{article}
\usepackage[utf8]{inputenc}
\usepackage[margin=1in]{geometry}
\usepackage{titling}

\renewcommand\maketitlehooka{\null\mbox{}\vfill}
\renewcommand\maketitlehookd{\vfill\null}

\begin{document}

\newcommand{\blank}{\vskip 3mm}
\setlength\parindent{0pt}
\renewcommand\thesection{\Alph{section}}

\setcounter{section}{2}
\section{Singing Low}

Devon enjoys singing, and wants to see how low he can sing. Devon begins at note $N$ ($1 \leq N \leq 10^6 $), and would like to sing note $0$. In one step, Devon may sing between $1$ and $K$ notes lower than his current note (so if $K$ is $3$ and he is on note $5$, he can sing either $3$, $2$, or $1$ notes lower than his current note, taking him to notes $2$, $3$, or $4$, respectively). However, Devon's note cannot decrease by any given amount more than once (so if he went from note $5$ to note $3$ in the previous example by going down $2$ notes, he would not be able to go to $1$ as this would again decrease by $2$ notes). Please help Devon calculate the smallest value of $K$ that will allow him to get to note $0$.

\blank
\textbf{INPUT FORMAT:}\\
The first line will contain the integer $ N $, the note that Devon begins on.
\blank
\textbf{OUTPUT FORMAT:}\\
The output should consist of one integer, $ K $, the lowest value which will allow Devon to reach note $ 0 $.
\blank
\textbf{SAMPLE INPUT:}
\begin{verbatim}
8
\end{verbatim}
\textbf{SAMPLE OUTPUT:}
\begin{verbatim}
4
\end{verbatim}
\end{document}
